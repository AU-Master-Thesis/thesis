\tikzstyle{equal below of} = [below=of #1.south]
\tikzstyle{equal right of} = [right=of #1.east]
\tikzstyle{equal left of} = [left=of #1.west]
\tikzstyle{equal above of} = [above=of #1.north]

% \def\bbx min (#1,#2,#3) max (#4,#5,#6) color #7 alpha #8; {
% 	%draw the bottom of the cube
%     \draw[cube,fill=#7,fill opacity=#8] (#1,#2,#3) -- (#1,#5,#3) -- (#4,#5,#3) -- (#4,#2,#3) -- cycle;
% 	%draw the back-right of the cube
% 	\draw[cube,fill=#7,fill opacity=#8] (#1,#2,#3) -- (#1,#5,#3) -- (#1,#5,#6) -- (#1,#2,#6) -- cycle;
% 	%draw the back-left of the cube
% 	\draw[cube,fill=#7,fill opacity=#8] (#1,#2,#3) -- (#4,#2,#3) -- (#4,#2,#6) -- (#1,#2,#6) -- cycle;
% 	%draw the front-right of the cube
% 	\draw[cube,fill=#7,fill opacity=#8] (#4,#2,#3) -- (#4,#5,#3) -- (#4,#5,#6) -- (#4,#2,#6) -- cycle;
% 	%draw the front-left of the cube
%     \draw[cube,fill=#7,fill opacity=#8] (#1,#5,#3) -- (#4,#5,#3) -- (#4,#5,#6) -- (#1,#5,#6) -- cycle;
% 	%draw the top of the cube
% 	\draw[cube,fill=#7,fill opacity=#8] (#1,#2,#6) -- (#1,#5,#6) -- (#4,#5,#6) -- (#4,#2,#6) -- cycle;
% }
\def\bbx[#1] min (#2,#3,#4) max (#5,#6,#7); {
%draw the bottom of the cube
\draw[cube,#1] (#2,#3,#4) -- (#2,#6,#4) -- (#5,#6,#4) -- (#5,#3,#4) -- cycle;
%draw the back-right of the cube
\draw[cube,#1] (#2,#3,#4) -- (#2,#6,#4) -- (#2,#6,#7) -- (#2,#3,#7) -- cycle;
%draw the back-left of the cube
\draw[cube,#1] (#2,#3,#4) -- (#5,#3,#4) -- (#5,#3,#7) -- (#2,#3,#7) -- cycle;
%draw the front-right of the cube
\draw[cube,#1] (#5,#3,#4) -- (#5,#6,#4) -- (#5,#6,#7) -- (#5,#3,#7) -- cycle;
%draw the front-left of the cube
\draw[cube,#1] (#2,#6,#4) -- (#5,#6,#4) -- (#5,#6,#7) -- (#2,#6,#7) -- cycle;
%draw the top of the cube
\draw[cube,#1] (#2,#3,#7) -- (#2,#6,#7) -- (#5,#6,#7) -- (#5,#3,#7) -- cycle;
}

\def\voxel[#1] center (#2,#3,#4) resolution #5; {
		\bbx[#1,fill opacity=0.35,draw opacity=1,line width=0.25,draw=black] min (#2-#5/2, #3-#5/2, #4-#5/2) max (#2+#5/2, #3+#5/2, #4+#5/2);
	}

\def\volumetricbox at (#1,#2) with dimensions (#3,#4,#5) color #6 alpha #7; {
		\def\bposx{#1}
		\def\bposy{#2}
		\def\bwidth{#3};
		\def\bheigth{#4};
		\def\bdepth{#5}

		\draw[draw=black] (\bposx,\bposy) -- ++(\bdepth,\bdepth) -- ++(0,\bheigth);
		\draw[draw=black] (\bposx+\bdepth,\bposy+\bdepth) -- ++(\bwidth,0);

		\draw[draw=black,fill opacity=#7,fill=#6] (\bposx,\bposy) rectangle +(\bwidth,\bheigth);

		\draw[draw=black,fill opacity=#7,fill=#6] (\bposx+\bwidth,\bposy) -- (\bposx+\bwidth+\bdepth,\bposy+\bdepth) -- (\bposx+\bwidth+\bdepth,\bposy+\bheigth+\bdepth) -- (\bposx+\bwidth,\bposy+\bheigth);

		\draw[draw=black,fill opacity=#7,fill=#6] (\bposx+\bwidth,\bposy+\bheigth) -- (\bposx+\bwidth+\bdepth,\bposy+\bheigth+\bdepth) -- (\bposx+\bdepth,\bposy+\bheigth+\bdepth) -- (\bposx,\bposy+\bheigth);
	}


\def\centerarc[#1](#2)(#3:#4:#5)% Syntax: [draw options] (center) (initial angle:final angle:radius)
{ \draw[#1] ($(#2)+({#5*cos(#3)},{#5*sin(#3)})$) arc (#3:#4:#5); }



%%
\NewDocumentCommand\Cycle{O{} m m m O{} m}{%
% [opt arg cycle]{Node}{Angle}{Node size}[opt arg arch node]{cycle size}
\draw[#1](#2.{#3+asin(#6/(#4*1.41))}) arc (180+#3-45:180+#3-45-270:#6/2) #5;
}
%%

\makeatletter
\def\vectornormalised#1#2{%
	\begingroup
	\coordinate (tempa) at (#1);
	\coordinate (tempb) at (#2);
	\pgfpointdiff{\pgfpointanchor{tempa}{center}}%
	{\pgfpointanchor{tempb}{center}}%
	\pgfpointnormalised{}
	\pgf@xa=28.45274\pgf@x%
	\pgf@ya=28.45274\pgf@y%
	\draw[red,->] (#1)-- +(\the\pgf@xa,\the\pgf@ya) coordinate (tempc);
	\endgroup
}

\tikzset{norm/.style={to path={%
					\pgfextra{\vectornormalised{\tikztostart}{\tikztotarget}} (tempc) -- (\tikztotarget) \tikztonodes
				}}}
\makeatother
